\documentclass{article}
\usepackage[margin=20mm]{geometry}
\usepackage{amsmath}
\usepackage{cleveref}
\usepackage{graphicx}
\usepackage{subcaption}

\title{My cool paper}
\author{John Doe}

\begin{document}

\maketitle

\section{Introduction}

This document is simply a template for what could be a scientific document.

\section{Trajectory}

Let us compute the trajectory of an object thrown with speed $v_0$ at an angle $\alpha$.

We suppose that the initial position of the object is at $(x_0, y_0)$.


We first consider the acceleration of the object.
In our case, there is only the gravitational force:

\begin{equation}\label{eq:acc}
    \begin{cases}
        \ddot{x}(t) &= 0 \\
        \ddot{y}(t) &= -g
    \end{cases}
\end{equation}

We then integrate \cref{eq:acc} and use the initial condition on the speed of the object:

\begin{equation}\label{eq:speed}
    \begin{cases}
        \dot{x}(t) &= v_0 \times \cos \alpha \\
        \dot{y}(t) &= -g \times t + v_0 \times \sin \alpha
    \end{cases}
\end{equation}

Finally, we integrate \cref{eq:speed} to find the formulas for the position of the object through time:

\begin{equation}
    \begin{cases}
        x(t) &= v_0 \times t \times \cos \alpha + x_0 \\
        y(t) &= -\frac{1}{2} \times g  \times t^2 + v_0 \times t \times \sin \alpha + y_0
    \end{cases}
\end{equation}

\section{Simulation}

We implemented a simple simulator, and we will explore the trajectory of the object based on the initial conditions, namely $v_0$ and $\alpha$.


\begin{figure}
    \centering
    \begin{subfigure}{0.45\textwidth}
        \includegraphics[width=\textwidth]{figs/v0.pdf}
        \caption{$v_0$}
    \end{subfigure}
    \begin{subfigure}{0.45\textwidth}
        \includegraphics[width=\textwidth]{figs/alpha.pdf}
        \caption{$\alpha$}
    \end{subfigure}
    \caption{plop}
\end{figure}


\end{document}
